\documentclass[a4paper, 11pt]{article}
\usepackage[slovene]{babel}
\usepackage[utf8]{inputenc}
\usepackage[T1]{fontenc}
\usepackage{amsfonts,amsmath,amssymb}
\usepackage{amsthm}
\usepackage{amsmath}
\usepackage{amssymb}

\begin{document}

\thispagestyle{empty}
\begin{center}
\begin{minipage}{0.75\linewidth}
    \centering
    {\Large Univerza v Ljubljani \\ Fakulteta za matematiko in fiziko}
    \\
    \vspace{7cm}

    {\uppercase{\Large \textbf{Najcenej"se prirejanje v ravnini}}} \\ Finančni praktikum \\
    \vspace{3cm}

    Avtorici:\\
    {\Large Iza Čebulj, Barbara Pal\par}
    \vspace{7cm}

    {\Large Ljubljana, 2022}
\end{minipage}
\end{center}

\newpage
 
\section{Navodilo}

Naj bo $P$ množica z $2n$ točkami na ravnini. Najcenejše prirejanje za $P$ je množica daljic z minimalno skupno dolžino, za katere velja da je vsaka točka iz množice $P$ končna točka natanko ene daljice. \\
Formuliraj problem kot CLP. Preko eksperimentov poišči pričakovano vrednost celotne dolžine najkrajšega ujemanja, ko so točke izbrane naključno v:
\begin{itemize}
    \item enotskem kvadratu,
    \item enotskem krogu,
    \item enakostraničnem trikotniku, \dots
\end{itemize}
Ugotovi, ali se vrednost z večanjem $n$ povečuje ali zmanjšuje. \\
Obravnavaj tudi problem, ko je množica $P$ sestavljena iz $n$ rdečih in $n$ modrih točk in so končne točke daljic različnih barv - dvobarvno najcenejše prirejanje. 
Primerjaj vrednost v primeru, ko točke ločimo po barvah in v primeru, ko jih ne.  

\section{Opis problema}

Množici $P$ z $2n$ točkami lahko priredimo neusmerjen graf $G(P),$ oziroma samo $G.$
Množica vozlišč grafa $G$ je kar množica $P,$ množica povezav v grafu $E$ pa so neurejeni pari $(u,v),$
za katere velja $u,~v \in P$ in $u \neq v.$ 
Cena povezave je razdalja $d(u,v)$ med vozliščema $u$ in $v.$
Prirejanje na grafu $G$ oziroma na množici $P$ je množica povezav $M,$ 
za katero velja, da nobeno vozlišče v $P$ ne sovpada z več kot eno povezavo v $M.$
Popolno prirejanje v $P$ je tako prirejanje $M,$ kjer vsako vozlišče sovpada z natanko eno povezavo v $M.$
Velikost popolnega prirejanja je $n.$ 
Ceno prirejanja definiramo kot $\sum_{(u,v) \in M} d(u,v),$ kar je vsota cen vseh povezav v $M.$

Naša naloga pri problemu najcenejšega prirejanja v ravnini - \textit{shortest matching in the plane} - je poiskati
popolno prirejanje množice $P$ z najnižjo ceno.

Ta problem ima uporabno vrednost v operacijskih raziskavah, prepoznavanju vzorcev in statistiki.
Uporabljajo ga pri določanju učinkovitega gibanja mehanskih risalnikov, kar je poseben primer problema kitajskega poštarja.
Iskanje najcenejšega prirejanja in njemu podobni problemi so rešljivi v polinomskem času z najbolj osnovnim algoritmom, 
seveda pa se pojavlja vprašanje, ali algoritme lahko še izboljšamo.

Drugi del naloge se v strokovni literaturi pojavlja pod imenom Evklidski dvodelni problem ujemanja \emph{(ang. The Euclidean Bipartite Matching Problem, EBM)}.
Množico $P$ sestavljata množica $n$ rdečih točk, $R,$ in množica $n$ modrih točk $B,$ $P=R \cup B.$
V tem primeru je $G(P,E)$ dvodelen graf z lastnostjo, da med dvema točkama obstaja povezava, če in samo če sta različnih barv.
Cene povezav $(u,v)$ so prav tako razdalje med vozlišči, $d(u,v).$

\section{Načrt dela}
V nadaljevanju bova najprej problem zapisali kot celoštevilski linearni program, nato pa bova še v programskem jeziku \textit{Python} in s pomočjo \textit{CoCalc-a} zapisali algoritem in ga preizkusili za različne $n$ ter pri večkratnih naključnih izbirah točk v enotskih likih.
Pri tem naju bo zanimala pričakovana vrednost najcenejšega popolnega prirejanja, torej celotna cena oziroma dolžina povezav, ki so v množici $M,$ med točkami, ki se nahajajo v enotskih likih.
Ugotovili bova tudi, ali se ta pričakovana vrednost zmanjšuje ali povečuje z večanjem števila $n,$ ter kakšna je v dvobarvnem grafu ter kakšno je trajanje algoritma - če deluje v linearnem, polinomskem ali eksponentnem času.
Končno pa bova rezultate eksperimentov prikazali z vizualizacijami.

\end{document}