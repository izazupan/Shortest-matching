\documentclass[a4paper, 11pt]{article}
\usepackage[slovene]{babel}
\usepackage[utf8]{inputenc}
\usepackage[T1]{fontenc}
\usepackage{amsfonts,amsmath,amssymb}
\usepackage{amsthm}
\usepackage{amsmath}
\usepackage{amssymb}

\begin{document}

\thispagestyle{empty}
\begin{center}
\begin{minipage}{0.75\linewidth}
    \centering
    {\Large Univerza v Ljubljani \\ Fakulteta za matematiko in fiziko}
    \\
    \vspace{7cm}

    {\uppercase{\Large \textbf{Shortest matching in the plane}}} \\ Finančni praktikum \\
    \vspace{3cm}

    Avtorici:\\
    {\Large Iza Čebulj, Barbara Pal\par}
    \vspace{7cm}

    {\Large Ljubljana, 2022}
\end{minipage}
\end{center}

\newpage
 
\section{Navodilo}

Naj bo $P$ množica z $2n$ točkami na ravnini. Najkrajše ujemanje za $P$ je množica segmentov z minimalno skupno dolžino, za katere velja da je vsaka točka iz množice $P$ končna točka natanko enemu segmentu. \\
Formuliraj problem kot CLP. Preko eksperimentov najdi pričakovano vrednost skupne dolžine najkrajšega ujemanja, ko so točke izbrane naključno v:
\begin{itemize}
    \item enotskem kvadratu,
    \item enotskem krogu,
    \item enakostraničnem trikotniku, \dots
\end{itemize}
Ugotovi ali se vrednost z večanjem $n$ povečuje ali zmanjšuje. \\
Obravnavaj tudi problem, ko je množica $P$ sestavljena iz $n$ rdečih in $n$ modrih točk in moramo ustvariti segmente katerih končne točke so različnih barv. Primerjaj vrednost v primeru ko točke ločimo po barvah in v primeru ko jih ne.  

\section{Načrt dela}

\end{document}